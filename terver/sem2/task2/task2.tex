\documentclass[11pt]{article}
\usepackage[utf8]{inputenc}
\usepackage[russian]{babel}
\usepackage{amsmath}
\usepackage{enumitem}
\usepackage{tabulary}
\usepackage[left=2cm, right=2cm, top=2cm]{geometry}

\begin{document}
\raggedright
\begin{sloppypar}
\section*{№1}
Найдем матожидание случайной величины $\xi$, равной количеству опечаток на некоторой странице.

\begin{equation}
  E\xi = \sum_{i = 1}^{50} E\xi_i = \sum_{i = 1}^{50} \frac{1}{500} = \frac{1}{10}
\end{equation}

...где $\xi_i$ - случайная величина, равная 1, если $i$-я по счету опечатка оказалась на выбранной странице, и 0 в противном случае.

\begin{enumerate}
  \item $P(\xi \geq 3) \leq \frac{E\xi}{3} = \frac{1}{30}$
  \item $P(\xi = 0) = 1 - P(\xi \geq 1) \geq \frac{9}{10}$
\end{enumerate}
  
\section*{№2}
Посчитаем дисперсию $\xi_1$, равной количеству очков, выпавших при первом броске.

\begin{equation}
  D\xi_1 = E((\xi - E\xi)^2) = \frac{1}{6}((1 - 3.5)^2 + (2 - 3.5)^2 + (3 - 3.5)^2 + (4 - 3.5)^2 + (5 - 3.5)^2 + (6 - 3.5)^2) = \frac{35}{12}
\end{equation}

Теперь оценим искомую вероятность используя закон больших чисел.

\begin{equation}
  P(|\frac{\sum_{i = 1}^{1000} \xi_i}{n} - E\xi_1| \leq 0.3) \geq 1 - P(|...| \geq 0.3) \geq 1 - \frac{D\xi_1}{1000 * 0.3^2} \approx 0.9676
\end{equation}

Таким образом, искомая вероятность не меньше чем 0.9676.

\section*{№3}
Уточним условие задачи. Судя по всему, количество кораблей равно количеству коммерсантов, а в фонд отправляются не 6\% прибыли, а 6\% от всей суммы убытков при захвате корабля. Тогда необходимо решить такую задачу: найти наименьшее n, при котором:

\begin{equation}
  P(\xi > 0.06n) \leq 0.05
\end{equation}

...где $\xi$ равна количеству захваченных кораблей. Из неравенства Маркова:

\begin{equation}
  P(\xi > 0.06n) \leq P(\xi \geq 0.06n) \leq \frac{E\xi}{0.06n} = \frac{5}{6n} \leq 0.05
\end{equation}

Из последнего неравенства получаем:

\begin{equation}
  n \geq \frac{50}{3} \approx 16.6
\end{equation}

Значит, наименьшее количество коммерсантов равно 17.

\section*{№4}
Пусть $\xi$ равна количеству ошибок в слове. Используя неравенство Маркова и свойства o-малых, получаем:

\begin{equation}
  P(\xi \geq 1) \leq E\xi = \sum_{i = 1}^n E\xi_i = \sum_{i = 1}^n o(\frac{1}{n}) = o(\frac{1}{n})
\end{equation}

...где $\xi_i$ равна 1, если в $i$-ой позиции возникла ошибка и 0 в противном случае. Отсюда:

\begin{equation}
  \lim_{n \to \infty} P(\xi \geq 1) \leq \lim_{n \to \infty} o(\frac{1}{n}) = 0
\end{equation}

\end{sloppypar}
\end{document}